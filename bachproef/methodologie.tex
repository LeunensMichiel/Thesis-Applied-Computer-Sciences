%%=============================================================================
%% Methodologie
%%=============================================================================
\chapter{Methodologie}
\label{ch:methodologie}

%% TODO: Hoe ben je te werk gegaan? Verdeel je onderzoek in grote fasen, en
%% licht in elke fase toe welke stappen je gevolgd hebt. Verantwoord waarom je
%% op deze manier te werk gegaan bent. Je moet kunnen aantonen dat je de best
%% mogelijke manier toegepast hebt om een antwoord te vinden op de
%% onderzoeksvraag.

%% Even een one-pager (of half-pager) waar je je strategie uitlegt. Uitleggen dat je op basis van de stand van zake tools hebt bekeken (ref 2.5) en samen met de probleemstelling (ref 3) een Proof-of-concept zal uitwerken (ref 5) om te voldoen aan de vraag van kayzr.

In \textit{\nameref{ch:stand-van-zaken}} werd er onderzoek gedaan naar verschillende middleware en tools voor het monitoren van een Node.js applicatie die gebruikt maakt van Express. In \textit{\nameref{ch:kayzrsProblem}} werd dit document aangevuld met functionaliteiten en vereisten die Kayzr in de software graag wenst te zien. 

Nu dat al de nodige informatie is verzameld, kan het ontwerpen van de middleware toepassing van start gaan. Daarom zal volgens een bepaalde werkwijze een \textit{proof-of-concept} worden uitgewerkt, waarover zo dadelijk in hoofdstuk \ref{ch:proofOfConcept} meer uitleg zal worden gegeven. Daarna volgt er een uitleg van de verschillende functionaliteiten die de software bevat.

Tijdens het ontwerpen van de middleware zal echter niet enkel met de requirements van Kayzr rekening gehouden worden. Het is de bedoeling dat dit geen \textit{hardcoded} stuk software in hun backend wordt (software die niet aanpasbaar en distribueerbaar is), maar een abstracte, universele middleware die gedownloaded en geïnstalleerd kan worden via Node.js package manager. Daarom zal een verwijzing naar Github en npm alsook de \textit{README.md}, een bestand dat informatie bevat over alle andere bestanden in de software, worden toegevoegd als bijlage om te bewijzen dat de software effectief openbaar staat om in een eigen project te verwerken.

Tenslotte zal er in hoofdstuk \ref{ch:conclusie} worden geconcludeerd of de software effectief in de productie-backend van Kayzr wordt ingehaakt, hoe Kayzr de toekomst ziet, en of deze studie kan uitgebreid worden d.m.v. een verder onderzoek.  

