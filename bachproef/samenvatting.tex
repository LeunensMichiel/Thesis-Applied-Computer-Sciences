%%=============================================================================
%% Samenvatting
%%=============================================================================

% TODO: De "abstract" of samenvatting is een kernachtige (~ 1 blz. voor een
% thesis) synthese van het document.
%
% Deze aspecten moeten zeker aan bod komen:
% - Context: waarom is dit werk belangrijk?
% - Nood: waarom moest dit onderzocht worden?
% - Taak: wat heb je precies gedaan?
% - Object: wat staat in dit document geschreven?
% - Resultaat: wat was het resultaat?
% - Conclusie: wat is/zijn de belangrijkste conclusie(s)?
% - Perspectief: blijven er nog vragen open die in de toekomst nog kunnen
%    onderzocht worden? Wat is een mogelijk vervolg voor jouw onderzoek?
%
% LET OP! Een samenvatting is GEEN voorwoord!

%%---------- Nederlandse samenvatting -----------------------------------------
%
% TODO: Als je je bachelorproef in het Engels schrijft, moet je eerst een
% Nederlandse samenvatting invoegen. Haal daarvoor onderstaande code uit
% commentaar.
% Wie zijn bachelorproef in het Nederlands schrijft, kan dit negeren, de inhoud
% wordt niet in het document ingevoegd.

\IfLanguageName{english}{%
\selectlanguage{dutch}
\chapter*{Samenvatting}
\lipsum[1-4]
\selectlanguage{english}
}{}

%%---------- Samenvatting -----------------------------------------------------
% De samenvatting in de hoofdtaal van het document

\chapter*{\IfLanguageName{dutch}{Samenvatting}{Abstract}}

Dit onderzoek wordt gebruikt om te kijken of er goede methodieken en/of betere alternatieven zijn om processen te debuggen van Node.js en Express applicaties in 2019. Dit werk is in opdracht van het bedrijf Kayzr, dat nood heeft aan een verbeterd debug-proces. Met een beperkt budget zijn bestaande tools ofwel onbetaalbaar, ofwel veel te primitief, en zou er onderzoek moeten gedaan worden aan een goed alternatief dat het beste van beide werelden combineert. Namelijk een betaalbaar pakket dat minimum moet voldoen aan een specifiek aantal functionaliteiten.

In deze proef wordt er onderzoek gedaan naar het ecosysteem van de huidige Javascript scène, en uitleg verschaft over diens meest populaire framework \textit{Node.js}. De vereisten van Kayzr worden geanalyseerd en er zal gekeken worden hoe dit kan opgelost worden met huidige hulpmiddelen en software. Vervolgens zal er een stuk software geschreven worden dat Kayzr's problematiek moet oplossen, waarna deze zal getoetst worden aan de praktijk.

 Het resultaat werd positief ontvangen en werd meteen in de productieomgeving gezet. De software kon ook worden geabstraheerd waardoor het mogelijk werd deze tool te delen met het wereldwijde web. Hierdoor kunnen andere bedrijven en ontwikkelaars hier ook van gebruik maken. De zelfgeschreven tool voldeed aan alle vereisten en er werden zelfs extra functionaliteiten toegevoegd.
 
 Er kan geconcludeerd worden dat de productiesnelheid van Kayzr omhoog is gegaan, de efficiëntie van het debuggen is verbeterd en kostprijs niet is gestegen. Mede dankzij de open source natuur van de tool, kan deze tool verder onderzocht worden om te bekijken of deze nog uitgebreid kan worden met nieuwe functionaliteiten of de huidige verbeterd en geoptimaliseerd kunnen worden. Ook kan er nog gekeken worden of de hoge kostprijs van betalende oplossingen verantwoord is door ze te vergelijken met deze gratis tool, en ook of de tool kan gebruikt worden in bedrijven met een ander businessplan dan Kayzr. Als laatste kan er altijd gekeken worden of er nog betere debug-procesmiddelen bestaan aangezien het ecosysteem van het internet verandert met de dag. 

