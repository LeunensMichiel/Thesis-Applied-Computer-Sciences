%%=============================================================================
%% Voorwoord
%%=============================================================================

\chapter*{Woord vooraf}
\label{ch:voorwoord}

%% TODO:
%% Het voorwoord is het enige deel van de bachelorproef waar je vanuit je
%% eigen standpunt (``ik-vorm'') mag schrijven. Je kan hier bv. motiveren
%% waarom jij het onderwerp wil bespreken.
%% Vergeet ook niet te bedanken wie je geholpen/gesteund/... heeft

Backend debug-toepassingen zijn vaak slecht onderhouden, incompleet of gewoonweg verschrikkelijk duur. Daarom is deze bachelorproef gemaakt met als doel een open-source debug tool te schenken aan iedereen die een project in Node.js en Express ontwikkelt. Mede mogelijk gemaakt dankzij Kayzr, is het de bedoeling om deze tool online te zetten zodat deze effectief in de praktijk gebruikt kan worden. 

Graag bedank ik meneer Tom Antjon, voor zijn altijd paraat staan, snel mijn vragen te willen beantwoorden en mij een goed inzicht te geven in het werk dat nog gedaan moest worden.

Ook wil ik Kayzr bedanken, voor een onvergetelijke stage, waar mijn vaardigheden als developer \textit{en} als fotograaf enorm hebben kunnen groeien. Zonder hen had ik nooit deze bachelorproef kunnen verwezenlijken. De vrijheid die ik hier kreeg, apprecieerde ik dan ook ten zeerste. Maar, dat zou vooral niet gelukt zijn zonder de hulp van mijn copromotor, Michiel Cuvelier, aan wie ik honderden technische vragen kon stellen en die hij stuk voor stuk uitvoerig beantwoordde. 

Het voelt enorm overweldigend aan  om een eerste open-source tool gepubliceerd te hebben op npm en Github. Ik ben dan ook zeer trots op het resultaat, en hoop dat ik met de hulp van andere ontwikkelaars deze tool verder kan blijven ontwikkelen waardoor meerdere gebruikers hier gebruik van zouden kunnen maken. 

Als laatste wil ik ook mijn vader bedanken voor zijn taal-en communicatieadvies.

