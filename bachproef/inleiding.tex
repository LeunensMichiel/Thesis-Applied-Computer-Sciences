%%=============================================================================
%% Inleiding
%%=============================================================================

\chapter{Inleiding}
\label{ch:inleiding}

%De inleiding moet de lezer net genoeg informatie verschaffen om het onderwerp te begrijpen en in te zien waarom de onderzoeksvraag de moeite waard is om te onderzoeken. In de inleiding ga je literatuurverwijzingen beperken, zodat de tekst vlot leesbaar blijft. Je kan de inleiding verder onderverdelen in secties als dit de tekst verduidelijkt. Zaken die aan bod kunnen komen in de inleiding~\autocite{Pollefliet2011}:
%
%\begin{itemize}
%  \item context, achtergrond
%  \item afbakenen van het onderwerp
%  \item verantwoording van het onderwerp, methodologie
%  \item probleemstelling
%  \item onderzoeksdoelstelling
%  \item onderzoeksvraag
%  \item \ldots
%\end{itemize}

\section{Probleemstelling}
\label{sec:probleemstelling}

%Uit je probleemstelling moet duidelijk zijn dat je onderzoek een meerwaarde heeft voor een concrete doelgroep. De doelgroep moet goed gedefinieerd en afgelijnd zijn. Doelgroepen als ``bedrijven,'' ``KMO's,'' systeembeheerders, enz.~zijn nog te vaag. Als je een lijstje kan maken van de personen/organisaties die een meerwaarde zullen vinden in deze bachelorproef (dit is eigenlijk je steekproefkader), dan is dat een indicatie dat de doelgroep goed gedefinieerd is. Dit kan een enkel bedrijf zijn of zelfs één persoon (je co-promotor/opdrachtgever).

Het zoeken naar fouten in software kan een frustrerend, hectisch en vermoeiend proces zijn. Daarom zijn toepassingen die dat proces voor de ontwikkelaar kunnen automatiseren of beter analyseren altijd welkom. Kayzr, een Belgische start-up die zich bezighoudt met het mainstream maken en het organiseren van E-Sports in de Benelux, wenst zulke software zodat hun huidige productieproces versneld kan worden. 
\begin{wrapfigure}{l}{0.20\textwidth}
	\includegraphics[width=0.20\textwidth]{kayzr.png}
\end{wrapfigure} Het concrete probleem van Kayzr is dat zijn processen op verschillende plekken worden gemonitord. De tools zijn incompleet, en daardoor krijgt men een gefragmenteerde collectie aan logs, foutmeldingen, informatie, enzovoort.  Per API call moet men foutmeldingen controleren in de terminal, IP-adressen opzoeken in het Google Cloud platform en andere nuttige informatie staat dan weer eens ergens anders opgeslagen. Hierdoor duurt het lang om fouten te analyseren en tot een mogelijke oplossing te komen. Een degelijke tool zoals PM2, een monitoringtoepassing met veel meer functionaliteiten, meer dan dat Kayzr nodig heeft, vormt meteen een kost die ze niet kunnen verantwoorden. Aan \euro 50 per server per maand, loopt het budget voor een vijftigtal servers te hoog op. Deze kostprijs zou schalen zeer moeilijk of zelfs onmogelijk maken.

\section{Onderzoeksvraag}
\label{sec:onderzoeksvraag}

%Wees zo concreet mogelijk bij het formuleren van je onderzoeksvraag. Een onderzoeksvraag is trouwens iets waar nog niemand op dit moment een antwoord heeft (voor zover je kan nagaan). Het opzoeken van bestaande informatie (bv. ``welke tools bestaan er voor deze toepassing?'') is dus geen onderzoeksvraag. Je kan de onderzoeksvraag verder specifiëren in deelvragen. Bv.~als je onderzoek gaat over performantiemetingen, dan 
De onderzoeksvraag is dus als volgt: bestaat er een mogelijkheid om de monitoringtechnieken van Kayzr, en mogelijk nog andere bedrijven, op een goedkope manier te stroomlijnen en te verbeteren binnen een Node.js omgeving, zodat hun debugproces geoptimaliseerd kan worden? 


\section{Onderzoeksdoelstelling}
\label{sec:onderzoeksdoelstelling}

%Wat is het beoogde resultaat van je bachelorproef? Wat zijn de criteria voor succes? Beschrijf die zo concreet mogelijk.
Het onderzoek is geslaagd wanneer Kayzr tevreden is met de toepassing en er in zijn project gebruik van gaat maken. De gemiddelde tijd om een probleem op te lossen moet dalen en het proces moet efficiënter verlopen. De data die per proces worden verzameld moeten op een goede manier kunnen gevisualiseerd worden zodat deze ook voor andere doeleinden kunnen worden gehanteerd. Er moeten meer data ontleed kunnen worden dan ervoor, en de software moet voldoen aan alle \hyperref[sec:requirements]{opgesomde requirements.}


\section{Opzet van deze bachelorproef}
\label{sec:opzet-bachelorproef}

% Het is gebruikelijk aan het einde van de inleiding een overzicht te
% geven van de opbouw van de rest van de tekst. Deze sectie bevat al een aanzet
% die je kan aanvullen/aanpassen in functie van je eigen tekst.

Verder in dit document bevinden zich de volgende onderdelen:

In Hoofdstuk~\ref{ch:stand-van-zaken} volgt een stand van zaken over JavaScript, Node.js, Express en Express middleware. Ook wordt er onderzoek gedaan naar verschillende monitoringsoftware en middleware die relevant zijn in 2019. Ook wordt het huidige debug proces bestudeerd. Hoe verloopt dit? Wat kan er beter? Wat kan er sneller? Welke functies moet de nieuwe toepassing bevatten?

In Hoofdstuk~\ref{ch:methodologie} wordt de software geschreven volgens de requirements van Kayzr. Vervolgens wordt er nagegaan of die software daaraan voldoet, en kan er worden vergeleken met de oude processen.

% TODO: Vul hier aan voor je eigen hoofstukken, één of twee zinnen per hoofdstuk

In Hoofdstuk~\ref{ch:conclusie} wordt er gekeken of er een oplossing bestaat op deze onderzoeksvragen. We kijken of Kayzr deze gaat gebruiken naar de toekomst toe en wat er gaat gebeuren met het eigendom van dit softwareproject.

