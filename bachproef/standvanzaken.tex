\chapter{Stand van zaken}
\label{ch:stand-van-zaken}

% Tip: Begin elk hoofdstuk met een paragraaf inleiding die beschrijft hoe
% dit hoofdstuk past binnen het geheel van de bachelorproef. Geef in het
% bijzonder aan wat de link is met het vorige en volgende hoofdstuk.

% Pas na deze inleidende paragraaf komt de eerste sectiehoofding.
%
%Dit hoofdstuk bevat je literatuurstudie. De inhoud gaat verder op de inleiding, maar zal het onderwerp van de bachelorproef *diepgaand* uitspitten. De bedoeling is dat de lezer na lezing van dit hoofdstuk helemaal op de hoogte is van de huidige stand van zaken (state-of-the-art) in het onderzoeksdomein. Iemand die niet vertrouwd is met het onderwerp, weet er nu voldoende om de rest van het verhaal te kunnen volgen, zonder dat die er nog andere informatie moet over opzoeken \autocite{Pollefliet2011}.
%
%Je verwijst bij elke bewering die je doet, vakterm die je introduceert, enz. naar je bronnen. In \LaTeX{} kan dat met het commando \texttt{$\backslash${textcite\{\}}} of \texttt{$\backslash${autocite\{\}}}. Als argument van het commando geef je de ``sleutel'' van een ``record'' in een bibliografische databank in het Bib\TeX{}-formaat (een tekstbestand). Als je expliciet naar de auteur verwijst in de zin, gebruik je \texttt{$\backslash${}textcite\{\}}.
%Soms wil je de auteur niet expliciet vernoemen, dan gebruik je \texttt{$\backslash${}autocite\{\}}. In de volgende paragraaf een voorbeeld van elk.
%
%\textcite{Knuth1998} schreef een van de standaardwerken over sorteer- en zoekalgoritmen. Experten zijn het erover eens dat cloud computing een interessante opportuniteit vormen, zowel voor gebruikers als voor dienstverleners op vlak van informatietechnologie~\autocite{Creeger2009}.

%	EFFECTIEVE LITERATUURSTUDIE

Dit hoofdstuk bestaat uit een zeer uitgebreide literatuurstudie, waarin de kern van het probleem wordt opgesplitst in verschillende delen. In het eerste deel wordt er uitleg gegeven over Javascript en diens evolutie. Vervolgens volgt er een groot deel over de frameworks NodeJS en Express, erna bekijken we het belang van testen en monitoren van software en breiden we uit naar een analyse van zulke monitoringssoftware. In het tweede deel wordt er onderzoek gedaan naar de vraag van Kayzr. Wat wordt er verwacht, welk inzicht moet er gegeven worden, welke features moeten worden uitgewerkt. 

\section{Javascript}
\label{sec:javascript}
Javascript is een programmeertaal gemaakt voor het web, waarmee u statische webpagina's kan omzetten naar dynamische en interactieve websites. Doordat het een enorm krachtige scripttaal is dat speciaal werd ontwikkeld om de functionaliteiten van een doorsnee HTML/CSS-pagina uit te breiden, wordt het bijna onmogelijk om nog iets in te beelden dat niet geïmplementeerd kan worden m.b.v. Javascript \autocite{Javascript2019}. De taal is weakly-typed, functioneel, event-driven en dynamisch.

\subsection{Tijdlijn}
\label{sec:javascriptTimeline}

Javascript heeft echter een hele evolutie achter de rug. Ontstaan in mei 1995, wordt de taal na vele updates nog steeds dagdagelijks gebruikt en kan het wereldwijde web niet meer ingebeeld worden zonder. Brendan Eich, de auteur van de taal, werkte in 1995 samen met Netscape Communications, de makers van de eerste grote webbrowser genaamd Netscape Navigator, om een taal te implementeren in hun browser waar webontwikkelaars gebruik van zouden kunnen maken. Java, een zeer zware programmeertaal met tal van functionaliteiten was de eerste keuze van Netscape, maar Eich schreef uiteindelijk zijn eigen idee uit van een scripttaal in nog geen 10 dagen en overtuigde Netscape om de lichte, schaalbare en Java-complementerende-taal te adopteren \autocite{Rangpariya2019}. Javascript, toen onder de naam Mocha en vervolgens LiveScript, was geboren.

 
