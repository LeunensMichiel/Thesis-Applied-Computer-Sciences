% !TeX spellcheck = de_DE
%---------- Inleiding ---------------------------------------------------------

\section{Introductie} % The \section*{} command stops section numbering
\label{sec:introductie}

Node.js is een backend Javascript framework wiens populariteit in de afgelopen jaren hard is toegenomen. Ontwikkelaars genieten van verschillende voordelen. Het werkt asynchroon, het is makkelijk schaalbaar en het is zeer portabel. Doordat het Javascript is, kan elk besturingssysteem gebruik maken van de krachtige functies die Node.js te bieden heeft. Dit maakt het een uitstekend framework om webapplicaties te ontwikkelen. Node.js is echter niet makkelijk om te debuggen doordat het asynchroon is opgebouwd. Het toepassen van de juiste monitortechnieken kan de slaagkansen van een project echter goed verhogen, net als de levensduur van de applicatie. Op welke manieren kunnen we het monitoren van zulke applicaties aanpakken? Welke software en tools worden hiervoor gebruikt? Welke technieken worden het best toegepast? En kan er ook intern in het proces van Node.js gekeken worden om daaruit nuttige informatie te halen? En zijn al deze technieken drastisch veranderd sinds het framework werd uitgebracht in maart 2009? 

%---------- Stand van zaken ---------------------------------------------------

\section{State-of-the-art}
\label{sec:state-of-the-art}

Node.js is reeds een matuur framework. Zoals gezegd wordt in ~\autocite{Runtime2017}: Node.js is sterk afhankelijk van asynchrone en continue programmeerstijl. I/O-bewerkingen worden uitgevoerd door middel van oproepen naar asynchrone functies waarbij een callback moet worden doorgegeven om aan te geven hoe de berekening wordt voortgezet zodra de genoemde I/O-bewerking asynchroon is voltooid. Het Node.js executiemodel bestaat uit een hoofdgebeurtenislus die wordt uitgevoerd op een single-threaded proces.
Het is daarom niet makkelijk om deze soort frameworks te debuggen, en wordt uiteindelijk een uitdagende taak. Gelukkig zijn hiervoor dan ook weer verschillende hulpmiddelen en tools uitgebracht om dit te vereenvoudigen. Zoals hierboven vermeld in ~\autocite{Runtime2017}, kan asynchrone code subtiele bugs opleveren die niet meteen zichtbaar zijn.  Hier is nog niet echt onderzoek over gedaan. Waarin er honderden vergelijkende studies bestaan over de frameworks zelf en of Node.js een goede optie is, zijn er geen of amper studies over de beste manier om dit te debuggen en hoe men dit het best aanpakt.. ~\autocite{Runtime2017} vertelt ons meer over  het identificeren van schaalbaarheidsproblemen en het aanrijken van mogelijke oplossingen. Ze maken gebruik van parametrische uitdrukkingen voor runtime monitoring van Node.js toepassingen, maar ze geven toe dat dit nog maar de eerste stap is en dat hier nog meer onderzoek naar kan gedaan worden. Doordat hier nog niet veel over staat neergepend, leek me dit een zeer interessant onderwerp dat hand in hand gaat met mijn stageopdracht. Kayzr, mijn stagebedrijf, zoekt namelijk zelf goede manieren om hun Node.js applicaties goed te kunnen debuggen, en deze studie zou zeker een goede bijdrage kunnen zijn aan deze soort doelgroep. Mijn onderzoek zal zich onderscheiden door een goed overzicht te behouden in die ongedocumenteerde zee van beschikbare frameworks, tools en software, waardoor deze studie kan gebruikt worden als een guideline voor best-practices. 

% Voor literatuurverwijzingen zijn er twee belangrijke commando's:
% \autocite{KEY} => (Auteur, jaartal) Gebruik dit als de naam van de auteur
%   geen onderdeel is van de zin.
% \textcite{KEY} => Auteur (jaartal)  Gebruik dit als de auteursnaam wel een
%   functie heeft in de zin (bv. ``Uit onderzoek door Doll & Hill (1954) bleek
%   ...'')


%---------- Methodologie ------------------------------------------------------
\section{Methodologie}
\label{sec:methodologie}

Onderzoek doen naar de verschillende mogelijkheden van software en tools. Een steekproef maken van een aantal node-developers door hen te contacteren en via een enquete vragen welke tools zij gebruiken om hun Node.js applicaties te monitoren, uitgebracht tussen 2009 en 2019. We kunnen erna kijken welke tools beter performeren dan anderen dankzij het gebruik van de servers van Kayzr. We kunnen er uitgebreider onderzoeken door:

\begin{itemize}
	\item Monitoren van api calls volgens het aantal keren opgeroepen
	\item Monitoren van api calls volgens duratie tot een response gestuurd wordt
	\item Het gemak om errors op te slaan en later te debuggen/analyseren
	\item Monitoren van deze node process en hun taxatie op de server waar ze op draaien (CPU, RAM, netwerk…)
\end{itemize}

De tools voor de bovenstaande metingen te testen gaan uiteraard de te onderzoeken tools, en hulpprogramma's als taakbeheer zijn. We maken daarna gebruik van R Studio om deze metingen en enquetes om te zetten tot mooie data waaruit we natuurlijk een conclusie kunnen trekken. 

Ook kunnen we literatuuronderzoek doen naar Javascript van ES1 naar ES6, NodeJS en zijn asynchrone processen, monitoring en het monitoren van async processen. 

%---------- Verwachte resultaten ----------------------------------------------
\section{Verwachte resultaten}
\label{sec:verwachte_resultaten}

Ik vermoed dat de resultaten uiteen zullen lopen, en dat verschillende developers zich liever vasthouden aan hun eigen methodes. Ook zal er duidelijk een breuk zijn in ontwikkelaars die liever blijven met hun vertrouwde tools die al een aantal jaar meegaan, en developers die liever het nieuwste van het nieuwste wensen te gebruiken. Dit zal het overzichtelijkst zijn in een bar chart met op de x-as de tool en op de y-as het aantal developers.

Qua effectieve data van de overwogen tools zal elk waarschijnlijk zijn voor en nadelen hebben. Toch vermoed ik dat we er een effectieve winnaar gaan uit kunnen halen die over het algemeen goed scoort. Dit zullen we dan toetsen aan de literatuurstudie om zo best practices te bekomen.

%---------- Verwachte conclusies ----------------------------------------------
\section{Verwachte conclusies}
\label{sec:verwachte_conclusies}

De wereld van Javascript is nog in volle groei, maar is toch al een pak matuurder dan vroeger. We zien dat de tools beter zijn geworden. De concurrentie is enorm groot aangezien Node.js een enorm populair ontwikkelaarsplatform is. Tool X zal over het algemeen de beste tool zijn om te gebruiken. De manier van aanpak gaat variëren, maar uit de steekproef kunnen we dat toch kunnen veralgemenen. 