%==============================================================================
% Sjabloon onderzoeksvoorstel bachelorproef
%==============================================================================
% Gebaseerd op LaTeX-sjabloon ‘Stylish Article’ (zie voorstel.cls)
% Auteur: Jens Buysse, Bert Van Vreckem

\documentclass[fleqn,10pt]{voorstel}

%------------------------------------------------------------------------------
% Metadata over het voorstel
%------------------------------------------------------------------------------

\JournalInfo{HoGent Bedrijf en Organisatie}
\Archive{Bachelorproef 2018 - 2019} % Of: Onderzoekstechnieken

%---------- Titel & auteur ----------------------------------------------------

% TODO: geef werktitel van je eigen voorstel op
\PaperTitle{Monitoring mogelijkheden in NodeJS toepassingen voor feilloos debuggen in 2019.}
\PaperType{Onderzoeksvoorstel Bachelorproef} % Type document

% TODO: vul je eigen naam in als auteur, geef ook je emailadres mee!
\Authors{Michiel Leunens\textsuperscript{1}} % Authors
\CoPromotor{Michiel Cuvelier\textsuperscript{2} (KAYZR)}
\affiliation{\textbf{Contact:}
  \textsuperscript{1} \href{mailto:michiel.leunens.y7743@student.hogent.be}{michiel.leunens.y7743@student.hogent.be};
  \textsuperscript{2} \href{mailto:michiel@kayzr.com}{michiel@kayzr.com};
}

%---------- Abstract ----------------------------------------------------------

\Abstract{
	De wereld van Javascript is in volle groei. Honderden frameworks vervangen ondertussen verouderde frameworks, en velen worden elk jaar op hun beurt weer vervangen. Een van die frameworks die nu al even bestaat en er in uitblinkt, is Node.js. Als een backend Javascript framework met een uitstekende open-source community, goede ondersteuning, schaalbaarheid en snelle asynchrone bouwstenen, is dit de favoriet van vele backend webontwikkelaars. Het debuggen van zo'n framework, kan echter een ambetante zaak zijn. Kunnen die Javascript frameworks wel goed gemonitord worden? Deze studie omvat een totaal onderzoeken en uitwerkingen van de mogelijkheden om NodeJS toepassingen te monitoren. Dit enerzijds op gebied van bestaande tools en wat de mogelijkheden ervan zijn, alsook zelf een onderzoek doen naar wat er bekenen kan worden intern in het proces van NodeJS. De doelstelling van dit onderwerp zal de lezer helpen met het debuggen van NodeJS applicaties door te onderzoeken wat de best practices zijn. Uit een steekproef bestaande uit verschillende webontwikkelaars zal gekeken worden welke tools er het populairst zijn. Deze tools zullen dan aan verschillende testen onderdaan worden. Die hulpmiddelen gaan we toetsen in de praktijk aan de hand van het monitoren van API calls en andere testen. Ook zal een uitgebreide literatuurstudie aantonen welke methodieken er het best toegepast worden bij het analyseren en debuggen van zulke frameworks. Dit zal getoetst worden aan de praktijk. Er wordt verwacht dat uit de overvloed aan hulpmiddelen er een paar gaan uitspringen. Het zal variëren van software die al minstens vijf jaar meegaat tot software die nog geen jaar op de markt verschenen is. De conclusie zal ontwikkelaars een goed overzicht geven van best-practices en software die  het beste gebruikt worden. Natuurlijk zal dit onderzoek geen jaren standhouden, aangezien javascript in de komende jaren nog veel evoluties zal meemaken. Maar het is eerder de bedoeling om toch een tijdelijk goed overzicht te geven.
}

%---------- Onderzoeksdomein en sleutelwoorden --------------------------------
% TODO: Sleutelwoorden:
%
% Het eerste sleutelwoord beschrijft het onderzoeksdomein. Je kan kiezen uit
% deze lijst:
%
% - Mobiele applicatieontwikkeling
% - Webapplicatieontwikkeling
% - Applicatieontwikkeling (andere)
% - Systeembeheer
% - Netwerkbeheer
% - Mainframe
% - E-business
% - Databanken en big data
% - Machineleertechnieken en kunstmatige intelligentie
% - Andere (specifieer)
%
% De andere sleutelwoorden zijn vrij te kiezen

\Keywords{Onderzoeksdomein. Webapplicatieontwikkeling --- NodeJS --- Monitoren} % Keywords
\newcommand{\keywordname}{Sleutelwoorden} % Defines the keywords heading name

%---------- Titel, inhoud -----------------------------------------------------

\begin{document}

\flushbottom % Makes all text pages the same height
\maketitle % Print the title and abstract box
\tableofcontents % Print the contents section
\thispagestyle{empty} % Removes page numbering from the first page

%------------------------------------------------------------------------------
% Hoofdtekst
%------------------------------------------------------------------------------

% De hoofdtekst van het voorstel zit in een apart bestand, zodat het makkelijk
% kan opgenomen worden in de bijlagen van de bachelorproef zelf.
% !TeX spellcheck = de_DE
%---------- Inleiding ---------------------------------------------------------

\section{Introductie} % The \section*{} command stops section numbering
\label{sec:introductie}

NodeJS is een Javascript framework wiens populariteit in de afgelopen jaar hard is toegenomen. Ontwikkelaars genieten van verschillende voordelen. Het werkt asynchroon, het is makkelijk schaalbaar en het is zeer portabel. Doordat het Javascript is, kan elk besturingssysteem gebruik maken van de krachtige backendfuncties van NodeJS. Dit maakt het een uitstekend framework voor webapplicaties te ontwikkelen.  NodeJS is echter niet makkelijk om te monitoren doordat het asynchroon is opgebouwd. Het toepassen van de juiste technieken om te monitoren kan de slaagkansen van een project echter goed verhogen, net als de levenscyclus van de applicatie. Op welke manieren kunnen we het monitoren van zulke applicaties aanpakken? Welke software en tools worden hiervoor gebruikt? Welke technieken worden het best toegepast? En kan er ook intern in het proces van NodeJS gekeken worden en deze informatie toegepast worden? En zijn al deze technieken drastisch veranderd sinds het framework werd uitgebracht in maart 2009? 

%---------- Stand van zaken ---------------------------------------------------

\section{State-of-the-art}
\label{sec:state-of-the-art}



Hier beschrijf je de \emph{state-of-the-art} rondom je gekozen onderzoeksdomein. Dit kan bijvoorbeeld een literatuurstudie zijn. Je mag de titel van deze sectie ook aanpassen (literatuurstudie, stand van zaken, enz.). Zijn er al gelijkaardige onderzoeken gevoerd? Wat concluderen ze? Wat is het verschil met jouw onderzoek? Wat is de relevantie met jouw onderzoek?

Verwijs bij elke introductie van een term of bewering over het domein naar de vakliteratuur, bijvoorbeeld~\autocite{Doll1954}! Denk zeker goed na welke werken je refereert en waarom.

% Voor literatuurverwijzingen zijn er twee belangrijke commando's:
% \autocite{KEY} => (Auteur, jaartal) Gebruik dit als de naam van de auteur
%   geen onderdeel is van de zin.
% \textcite{KEY} => Auteur (jaartal)  Gebruik dit als de auteursnaam wel een
%   functie heeft in de zin (bv. ``Uit onderzoek door Doll & Hill (1954) bleek
%   ...'')


%---------- Methodologie ------------------------------------------------------
\section{Methodologie}
\label{sec:methodologie}

Onderzoek doen naar de verschillende mogelijkheden van software en tools. Een steekproef maken van een aantal node-developers en via een enquete vragen met welke tools zij hun NodeJS applicatie monitoren, uitgebracht tussen 2009 en 2019. We kunnen erna kijken welke tools beter performeren dan anderen dankzij het gebruik van de servers van Kayzr. We kunnen dit uitgebreider onderzoeken door:

\begin{itemize}
	\item Monitoren van api calls volgens het aantal keren opgeroepen
	\item Monitoren van api calls volens duratie tot een response gestuurd wordt
	\item Het gemak om errors op te slaan en later te debuggen/analyseren
	\item Monitoren van deze node process en hun taxatie op de server waar ze draaien (CPU, RAM, netwerk, …)
\end{itemize}

Hier beschrijf je hoe je van plan bent het onderzoek te voeren. Welke onderzoekstechniek ga je toepassen om elk van je onderzoeksvragen te beantwoorden? Gebruik je hiervoor experimenten, vragenlijsten, simulaties? Je beschrijft ook al welke tools je denkt hiervoor te gebruiken of te ontwikkelen. 

%---------- Verwachte resultaten ----------------------------------------------
\section{Verwachte resultaten}
\label{sec:verwachte_resultaten}

Er zal waarschijnlijk wel een tool uitspringen die op alle vlakken gemiddeld goed performeert. Dit zal wel de tool zijn die het meest populair is idk

Hier beschrijf je welke resultaten je verwacht. Als je metingen en simulaties uitvoert, kan je hier al mock-ups maken van de grafieken samen met de verwachte conclusies. Benoem zeker al je assen en de stukken van de grafiek die je gaat gebruiken. Dit zorgt ervoor dat je concreet weet hoe je je data gaat moeten structureren.

%---------- Verwachte conclusies ----------------------------------------------
\section{Verwachte conclusies}
\label{sec:verwachte_conclusies}

De wereld van Javascript is nog in volle groei, maar is toch al een pak matuurder dan vroeger. We zien dat de tools beter zijn geworden. De concurrentie is enorm groot aangezien NodeJS een enorm populair platform is. Daardoor zal het noodzakelijk zijn om een gepaste monitoringtechniek toe te passen adhv software die we reeds kennen en vertrouwen. 

Hier beschrijf je wat je verwacht uit je onderzoek, met de motivatie waarom. Het is \textbf{niet} erg indien uit je onderzoek andere resultaten en conclusies vloeien dan dat je hier beschrijft: het is dan juist interessant om te onderzoeken waarom jouw hypothesen niet overeenkomen met de resultaten.



%------------------------------------------------------------------------------
% Referentielijst
%------------------------------------------------------------------------------
% TODO: de gerefereerde werken moeten in BibTeX-bestand ``voorstel.bib''
% voorkomen. Gebruik JabRef om je bibliografie bij te houden en vergeet niet
% om compatibiliteit met Biber/BibLaTeX aan te zetten (File > Switch to
% BibLaTeX mode)

\phantomsection
\printbibliography[heading=bibintoc]

\end{document}
